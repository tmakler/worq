\documentclass[a4paper]{article}

\usepackage[english]{babel} 
\usepackage[utf8]{inputenc}
\usepackage{amsmath}
\usepackage{graphicx}
\usepackage{float}
\usepackage[colorinlistoftodos]{todonotes}
\usepackage{pst-all}
\usepackage{caption}
\usepackage{float}
\usepackage{fullpage}

\title{WORQ 7}

\author{Theo Makler}

\date{\today}

\begin{document}
\maketitle

\section{Problem B}

Given that $\frac{x}{2}=\frac{y}{4}$, Phillip claims that $\frac{x+1}{2}=\frac{y+1}{4}$ because he added the same value to both sides. is his claim correct? Why or why not?

As a start, I pluged in values for $x$ and $y$ that made the first equation true. I came up with $\frac{1}{2}=\frac{2}{4}$. If we use these values in the second equation, we get $\frac{1+1}{2}=\frac{2+1}{4}$, or $\frac{2}{2}=\frac{3}{4}$, which is not true. 

As a rule,

\section{Problem C}

Chose values for $a$ and $b$ so that $(4,-3)$, $(-2,b)$, and $(a,7)$ are collinear. Show that your points are collinear.

For these three points to be collinear, they need to be in a line. If we graph the current points on a graph, they would look like this:

\begin{figure}[h]
\centering
\begin{pspicture}(0,-0.5)(5,6)
\psset{unit=0.5}
\psaxes[labels=none]{->}(11,11)[$x$,-90][$y$,180]
\psaxes[labels=none]{->}(-11,-11)[$x$, -90][$y$, 180]
\psdot(4,-3)
\rput(4,-4){\psframebox*{$(4, -3)$}}
\psline{-}(-2,-11)(-2,11)
\rput(-2,-3){\psframebox*{$(-2,b)$}}
\psline{-}(-11,7)(11,7)
\rput(3,7){\psframebox*{$(a,7)$}}
\end{pspicture}
\end{figure}

This graph shows the point $(4.-3)$ and shows possible points for $-2,b)$ and $(a,7)$. If we draw a line from $(4,-3)$ that touches these points. The answer is b=2, a=-8.

\end{document}