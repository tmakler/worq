\documentclass[a4paper]{article}

\usepackage[english]{babel} 
\usepackage[utf8]{inputenc}
\usepackage{amsmath}
\usepackage{graphicx}
\usepackage{float}
\usepackage[colorinlistoftodos]{todonotes}
\usepackage{pst-all}
\usepackage{caption}
\usepackage{float}
\usepackage{fullpage}

\title{WORQ 16}

\author{Theo Makler}

\date{\today}

\begin{document}
\maketitle

\section{Problem A}

The dimensions $2 x 4 x 8$ create a rectangular prism with the same volume as a cube with side length $4$.

To solve this problem, I first solved for the volume of the cube. $4^2=64$, so I know that my rectangular prism has to have a volume of 64 cubic units. If I make one of my sides 4 units, then I know that the other two units product must equal 16 because $64 \div 4 = 16$. The dimensions 2 units and 8 units works because $2\times8=16$. These dimensions make the rectangular prism have the same volume as the cube.

\section{Problem C}



\end{document}