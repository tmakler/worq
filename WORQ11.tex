\documentclass[a4paper]{article}

\usepackage[english]{babel} 
\usepackage[utf8]{inputenc}
\usepackage{amsmath}
\usepackage{graphicx}
\usepackage{float}
\usepackage[colorinlistoftodos]{todonotes}
\usepackage{pst-all}
\usepackage{caption}
\usepackage{float}
\usepackage{fullpage}

\title{WORQ 11}

\author{Theo Makler}

\date{\today}

\begin{document}
\maketitle

\section{Problem A}

Estimate the area of the trapezoid given the square unit shown below. Explain how you determined your estimate.

To estimate the area,  measured the length of the square given, which was $\frac{1}{4}$" long on each side. From this, I measured the trapezoid's hight and length of the top and bottom sides. It was $2\frac{1}{2}$ units tall with the bottom being $10$ units and top $4$ units. After getting the measurements, I put them into the equation for the area of a trapezoid:
$$A=\frac{1}{2}h(b_{1}+b_{2})$$
$$A=\frac{1}{2}\times2\frac{1}{2}(10+4)$$
$$A=\frac{1}{2}\times2\frac{1}{2}\times14$$
$$A=\frac{1}{2}\times35$$
$$A=17\frac{1}{2}$$

From the equation for the area of a trapezoid, I found that the given trapezoid was about $17\frac{1}{2}$ units in area.

\section{Problem C}

Give the coordinates of the vertices of a non-square rectangle whose sides are not parallel to the axes. Explain why your vertices satisfy the conditions.

To make a rectangle whose sides are not parallel to the axes, I can make the sides have the slopes of 1 and $-1$. Also, to be a rectangle, two sides have to be longer than the other two sides. To solve this, I made the short sides go over 1 and up 1, and the long sides go over 2 and down 2. These made me pick the coordinates $A=(0,2)$, $B=(2,0)$, $C=(3,1)$ and $D=(1,3)$.

To show that these points are in a rectangle that satisfies the conditions, I can prove that there are two sets of parallel lines, there is a right angle, and that the short side is shorter than the long side. The equation for the parallel lines is this:

Line $AB$'s slope:
$$\frac{0-2}{2-0}=-1$$
Line $CD$'s slope:
$$\frac{3-1}{1-3}=-1$$
Line $AD$'s slope:
$$\frac{3-2}{1-0}=1$$
Line $BC$'s slope:
$$\frac{1-0}{3-2}=1$$

As these show, lines $AB$ and lines $CD$ are parallel, and lines $AD$ and $BC$ are parallel. This shows that two lines are perpendicular:

Line$AB$'s slope: $-1$
Line$AC$'s slope:$1$
$-1$ is the opposite reciprocal of $1$, making lines $AB$ and $AC$ perpendicular.

Now that I have shown that the figure is a rectangle, I need to show that it satisfies the conditions:

Length of line $AB$: 
$$a^2+b^2=c^2$$
$$2^2+2^2=c^2$$
$$4+4=c^2$$
$$8=c^2$$
$$\sqrt{8}=c$$

Length of line $AC$
$$a^2+b^2=c^2$$
$$1^2+1^2=c^2$$
$$1+1=c^2$$
$$2=c^2$$
$$\sqrt{2}=c$$

These equations show that line $AB$ is longer than line $AC$, making the rectangle fit the given requirements. The diagram is shown here:

\begin{figure}[h]
\centering
\begin{pspicture}(0,-0.5)(5,6)
\psset{unit=0.5}
\psaxes[labels=none]{->}(11,11)[$x$,-90][$y$,180]
\psline{-}(0,2)(2,0)
\psline{-}(3,1)(2,0)
\psline{-}(1,3)(3,1)
\psline{-}(1,3)(0,2)
\rput(-2,2){\psframebox*{$A(0,2)$}}
\rput(2,-1){\psframebox*{$B(2,0)$}}
\rput(5,1){\psframebox*{$C(3,1)$}}
\rput(2,3.5){\psframebox*{$D(1,3)$}}
\psdot(0,2)
\psdot(2,0)
\psdot(1,3)
\psdot(3,1)
\end{pspicture}
\end{figure}


\end{document}