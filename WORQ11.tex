\documentclass[a4paper]{article}

\usepackage[english]{babel} 
\usepackage[utf8]{inputenc}
\usepackage{amsmath}
\usepackage{graphicx}
\usepackage{float}
\usepackage[colorinlistoftodos]{todonotes}
\usepackage{pst-all}
\usepackage{caption}
\usepackage{float}
\usepackage{fullpage}

\title{WORQ 11}

\author{Theo Makler}

\date{\today}

\begin{document}
\maketitle

\section{Problem A}

Estimate the area of the trapezoid given the square unit shown below. Explain how you determined your estimate.

To estimate the area,  measured the length of the square given, which was $\frac{1}{4}$" long on each side. From this, I measured the trapezoid's hight and length of the top and bottom sides. It was $2\frac{1}{2}$ units tall with the bottom being $10$ units and top $4$ units. After getting the measurements, I put them into the equation for the area of a trapezoid:
$$A=\frac{1}{2}h(b_{1}+b_{2})$$
$$A=\frac{1}{2}\times2\frac{1}{2}(10+4)$$
$$A=\frac{1}{2}\times2\frac{1}{2}\times14$$
$$A=\frac{1}{2}\times35$$
$$A=17\frac{1}{2}$$

From the equation for the area of a trapezoid, I found that the given trapezoid was about $17\frac{1}{2}$ units in area.

\section{Problem C}

Give the coordinates of the vertices of an isosceles triangle whose axis of symmetry is not parallel to either axis. Show that your triangle is isosceles.

I know that if I make an isosceles triangle whose base has a slope of 1, the axis of symmetry will not be parallel to either axis. Because of this, I started with a base that went from $(1,1)$ to $(5,5)$. The axis of symmetry has to be perpendicular to the base (i.e. have a slope of $-1$) and has to pass through the midpoint, which is $(3,3)$. Using point-slope formula, the equation of the line is :

$$y-3=-1(x-3)$$
$$y-3=-x+3$$
$$y=-x+6$$

Any point along the axis of symmetry (except $(3,3)$) will make an isosceles triangle with $(1,1)$ and $(5,5)$. I chose $(0,6)$ because it is the y-intercept. 

\begin{figure}[h]
\centering
\begin{pspicture}(0,-0.5)(5,6)
\psset{unit=0.5}
\psaxes[labels=none]{->}(11,11)[$x$,-90][$y$,180]
\psline{-}(1,1)(5,5)
\psline[linestyle=dashed,dash=3pt 2pt](6,0)(-3,9)
\psline{-}(0,6)(1,1)
\psline{-}(0,6)(5,5)
\rput(-2,6){\psframebox*{$A(0,6)$}}
\rput(3,1){\psframebox*{$B(1,1)$}}
\rput(5,6){\psframebox*{$C(5,5)$}}
\end{pspicture}
\end{figure}

The answer for these points is $A=(0,6)$, $B=(1,1)$ and $C=(5,5)$. See diagram on next page.

\end{document}