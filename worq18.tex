\documentclass[a4paper]{article}

\usepackage[english]{babel} 
\usepackage[utf8]{inputenc}
\usepackage{amsmath}
\usepackage{graphicx}
\usepackage{float}
\usepackage[colorinlistoftodos]{todonotes}
\usepackage{pst-all}
\usepackage{caption}
\usepackage{float}
\usepackage{fullpage}

\title{WORQ 18}

\author{Theo Makler}

\date{\today}

\begin{document}
\maketitle

\section{Problem B}

The dimensions 3 units by 2 units make the rectangle have a perimeter that is five times greater than the length of one of the sides.

To solve this problem, I started by guessing and checking until I found the answer, which I proved using these equations. I assumed that the perimeter would be five times the length of the side with length two because it has the least value:

$$2\times(3+2)=5\times2$$
$$2\times5=5\times2$$
$$10=10$$

As you can see, the dimensions satisfy the conditions. After doing this problem, I saw the numbers and wondered if any multiples of those numbers could still satisfy the conditions. I wrote this equation to see if it was possible:

$$2\times(3x+2x)=5\times2x$$
$$2\times5x=5\times2x$$
$$10x=10x$$

\section{Problem C}

The points $(2,1)$, $(3,3)$, $(2,5)$, and $(1,3)$ create a rhombus without touching any of the axes.

I know that a rhombus has two equal sides, so as long as I make the two points at the end parallel and the two points at the sides parallel and halfway between the outside points, I will have a rhombus. The line that runs through the points at the sides must also be perpendicular to the line that runs through the points at the top and bottom. With these criteria in mind, I decided to start with the point $(2,1)$ because then I could create the rhombus without touching any of the sides. I eventually came up with this (figure on next page):

\begin{figure}[h]
\centering
\begin{pspicture}(-3,-3)(5,5)
\psset{unit=0.5}
\psaxes[labels=none]{<->}(0,0)(-1,-1)(5,6)[$x$,-90][$y$,180]
\psdot(1,3)
\psdot(2,5)
\psdot(3,3)
\psdot(2,1)
\psline{-}(1,3)(2,5)
\psline{-}(2,5)(3,3)
\psline{-}(3,3)(2,1)
\psline{-}(2,1)(1,3)
\end{pspicture}
\end{figure}

As this figure shows, the points $(2,1)$, $(3,3)$, $(2,5)$, and $(1,3)$ make a rhombus that does not touch the axes.

\end{document}