\documentclass[a4paper]{article}

\usepackage[english]{babel} 
\usepackage[utf8]{inputenc}
\usepackage{amsmath}
\usepackage{graphicx}
\usepackage{float}
\usepackage[colorinlistoftodos]{todonotes}
\usepackage{pst-all}
\usepackage{caption}
\usepackage{float}
\usepackage{fullpage}

\title{WORQ 19}

\author{Theo Makler}

\date{\today}

\begin{document}
\maketitle

\section{Problem B}
A trapezoid with three sides that are 1 unit long and one side that is 2 units long creates a quadrilateral with exactly 3 congruent sides.

When I started this problem, I knew that the quadrilateral had to be a trapezoid because if the figure has 3 congruent sides, then one side must be either shorter or longer. I made a quadrilateral with the top and sides all one unit long:

\begin{figure}[h]
\centering
\begin{pspicture}(.25,.25)(.25,.25)
\psset{unit=1}
\psdot(-.5,.85)
\psdot(.5,.85)
\psdot(-1,0)
\psdot(1,0)
\psline{-}(-.5,.85)(.5,.85)
\psline{-}(-1,0)(1,0)
\psline{-}(-.5,.85)(-1,0)
\psline{-}(.5,.85)(1,0)
\end{pspicture}
\end{figure}

I made the bottom base have a length of 2 because I know that 2 is greater than one:

\begin{figure}[h]
\centering
\begin{pspicture}(.25,1.5)(.25,.25)
\psset{unit=1.5}
\psdot(-.5,.85)
\psdot(.5,.85)
\psdot(-1,0)
\psdot(1,0)
\psline{-}(-.5,.85)(.5,.85)
\psline{-}(-1,0)(1,0)
\psline{-}(-.5,.85)(-1,0)
\psline{-}(.5,.85)(1,0)
\psline[linestyle=dashed,dash=3pt 2pt](-.5,.85)(-.5,0)
\psline[linestyle=dashed,dash=3pt 2pt](.5,.85)(.5,0)
\rput(-1,.5){\psframebox*{$1$}}
\rput(1,.5){\psframebox*{$1$}}
\rput(0,1){\psframebox*{$1$}}
\rput(0,-.15){\psframebox*{$2$}}
\rput(0,1.2){\psframebox*{$$}}
\rput(0,1.2){\psframebox*{$$}}
\end{pspicture}
\end{figure}

As you can see, this is a quadrilateral with exactly 3 congruent sides.

\section{Problem C}

The equation $-\sqrt{x^6}=\frac{3^4+3^3+540}{3}$ makes $x=-6$, the equation $x=\frac{x}{3}\times \frac{\sqrt{421201}}{649}$ has no solutions, and the equation $\frac{\sqrt{266256}}{\sqrt{931225}}\times\frac{\sqrt{931225}}{\sqrt{266256}}=\frac{(x+1)(x-1)}{x^2-1}$ has infinite solutions.

We can prove these problems by simplifying them. For the first one, we can just substitute $x$ for $-6$ like this:

\begin{align*}
-\sqrt{x^6}&=\frac{3^4+3^3+540}{3} \\
-\sqrt{-6^6}&=\frac{3^4+3^3+540}{3} \\
-(-216)&=\frac{648}{3} \\
216&=216 
\end{align*}

As this shows, the problem $-\sqrt{x^6}=-\frac{3^4+3^3+540}{3}$ makes $x=-6$ We can simplify the second problem to prove it has no solutions:

\begin{align*}
x&=\frac{x}{3}\times \frac{\sqrt{421201}}{649} \\
x&=\frac{x}{3}\times\frac{649}{649}\\
x&\ne\frac{x}{3} 
\end{align*}

As this shows, the problem $x=\frac{x}{3}\times \frac{\sqrt{421201}}{649}$ does not have any solutions. Finally, the last problem has infinite solutions:

\begin{align*}
\frac{\sqrt{266256}}{\sqrt{931225}}\times\frac{\sqrt{931225}}{\sqrt{266256}}&=\frac{(x+1)(x-1)}{x^2-1} \\
1&=\frac{x^2-1}{x^2-1}\\
1&=1 
\end{align*}

As these equations show, the problem $\frac{\sqrt{266256}}{\sqrt{931225}}\times\frac{\sqrt{931225}}{\sqrt{266256}}=\frac{(x+1)(x-1)}{x^2-1}$ has infinite solutions.

\section{Problem D}

The points $(1,1)$, $(4,1)$, $(3,2)$, and $(2,2)$ make an isosceles trapezoid that does not touch the axes.

The definition of an isosceles trapezoid by Wikipedia is "In Euclidean geometry, an isosceles trapezoid (isosceles trapezium in British English) is a convex quadrilateral with a line of symmetry bisecting one pair of opposite sides, making it automatically a trapezoid." This means that the trapezoid must be symetrical. I first plotted my points on a graph like this: 

\begin{figure}[h]
\centering
\begin{pspicture}(0,3.5)(0,0)
\psset{unit=0.5}
\psaxes[labels=none]{->}(7,7)[$x$,-90][$y$,180]
\psline{-}(1,1)(4,1)
\psline{-}(4,1)(3,2)
\psline{-}(3,2)(2,2)
\psline{-}(2,2)(1,1)
\psdot(1,1)
\psdot(4,1)
\psdot(3,2)
\psdot(2,2)
\psline[linestyle=dashed,dash=3pt 2pt](2.5,0)(2.5,7)
\end{pspicture}
\end{figure}

The line of symmetry would be along the dotted line. This figure shows an isosceles trapezoid with the points $(1,1)$, $(4,1)$, $(3,2)$, and $(2,2)$ that do not touch the axes.

\end{document}