\documentclass[a4paper]{article}

\usepackage[english]{babel} 
\usepackage[utf8]{inputenc}
\usepackage{amsmath}
\usepackage{graphicx}
\usepackage{float}
\usepackage[colorinlistoftodos]{todonotes}
\usepackage{pst-all}
\usepackage{caption}
\usepackage{float}
\usepackage{fullpage}

\title{WORQ 13}

\author{Theo Makler}

\date{\today}

\begin{document}
\maketitle

\section{Problem B}

Casey claims he has divided the rectangle below into four equal areas. Terrell disagrees. Who is correct and why?

\begin{figure}[h]
\centering
\begin{pspicture}(0,-0.5)(2,2)
\psset{unit=0.5}
\psaxes[labels=none]{->}(11,4)[$x$,-90][$y$,180]
\psline{-}(1,1)(9,1)
\psline{-}(9,1)(9,3)
\psline{-}(9,3)(1,3)
\psline{-}(1,3)(1,1)
\psline{-}(9,3)(1,1)
\psline{-}(9,1)(1,3)
\end{pspicture}
\end{figure}

To solve this problem, I labeled the lengths of the sides in the following way:

\begin{figure}[h]
\centering
\begin{pspicture}(0,-0.5)(2,2)
\psset{unit=0.5}
\psaxes[labels=none]{->}(11,4)[$x$,-90][$y$,180]
\psline{-}(1,1)(9,1)
\psline{-}(9,1)(9,3)
\psline{-}(9,3)(1,3)
\psline{-}(1,3)(1,1)
\psline{-}(9,3)(1,1)
\psline{-}(9,1)(1,3)
\rput(5,3.5){\psframebox*{$l$}}
\rput(9.5,2){\psframebox*{$w$}}
\end{pspicture}
\end{figure}

The length $l$ represents the length of the top and bottom sides, and the length $w$ represents the length of the two sides. With these lengths, we can write a series of equations to prove that the triangles are the same:

$$\frac{1}{2} \left(l \times \frac{1}{2}w\right) = \frac{1}{2} \left(w \times \frac{1}{2}l\right)$$
$$l \times \frac{1}{2}w = w \times \frac{1}{2}l$$
$$\frac{1}{2}\times l \times w = \frac{1}{2}\times l \times w$$

As these equations show, the two sets of triangles both have exactly the same area.

\section{Problem C}

The vertices of triangle $ABC$ have coordinates $A(-2,1)$, $B(4,3)$, and $C(6,-1)$. If $D$ is the origin, find the coordinates of $E$ and $F$ so that $ABC=DEF$. Justify your response. 

To solve this problem, I first graphed the known points like so:

\begin{figure}[h]
\centering
\begin{pspicture}(-3,-3)(5,5)
\psset{unit=0.5}
\psaxes[labels=none]{<->}(0,0)(-6,-6)(7,7)[$x$,-90][$y$,180]
\psline{-}(-2,1)(4,3)
\psline{-}(4,3)(6,-1)
\psline{-}(6,-1)(-2,1)
\psdot(0,0)
\rput(-2.5,1){\psframebox*{$A$}}
\rput(4,3.5){\psframebox*{$B$}}
\rput(6.5,-1){\psframebox*{$C$}}
\rput(-0.6,-0.6){\psframebox*{$D$}}
\end{pspicture}
\end{figure}

To create triangle $DEF$, lets make point $D$ be in the same place as point $A$, point $E$ in the same place as point $B$, and point $F$ the same as point $C$. From point $A$, point $B$ is over 6 units and up 2 units. From point $B$, point $C$ is 
down 4 and across 2. If we use these in our second triangle, we find that point $E$ is at $(6,2)$, and point $F$ is at $(8,-2)$. If we graph the points, it looks like this:

\begin{figure}[h]
\centering
\begin{pspicture}(-3,-3)(5,5)
\psset{unit=0.5}
\psaxes[labels=none]{<->}(0,0)(-6,-6)(7,7)[$x$,-90][$y$,180]
\psline{-}(-2,1)(4,3)
\psline{-}(4,3)(6,-1)
\psline{-}(6,-1)(-2,1)
\psdot(0,0)
\psdot(6,2)
\psdot(8,-2)
\rput(-2.5,1){\psframebox*{$A$}}
\rput(4,3.5){\psframebox*{$B$}}
\rput(6.5,-1){\psframebox*{$C$}}
\rput(-0.6,-0.6){\psframebox*{$D$}}
\psline{-}(0,0)(6,2)
\psline{-}(6,2)(8,-2)
\psline{-}(8,-2)(0,0)
\rput(6,2.6){\psframebox*{$E$}}
\rput(8.6,-2){\psframebox*{$F$}}
\end{pspicture}
\end{figure}

As these figures show, the points for the new triangle are $D(0,0)$, $E(6,2)$, and $F(8,-2)$. Diagrams on the next page.

\end{document}