\documentclass[a4paper]{article}

\usepackage[english]{babel} 
\usepackage[utf8]{inputenc}
\usepackage{amsmath}
\usepackage{graphicx}
\usepackage{float}
\usepackage[colorinlistoftodos]{todonotes}
\usepackage{pst-all}
\usepackage{caption}
\usepackage{float}
\usepackage{fullpage}

\title{WORQ 12}

\author{Theo Makler}

\date{\today}

\begin{document}
\maketitle

\section{Problem A}

Draw and give the dimensions of a rectangle and a triangle that have the same area. Explain why the two figures have the same area.

To start, I made a rectangle with an area of 16 square units. It looks like this:

\begin{figure}[h]
\centering
\begin{pspicture}(0,-0.5)(5,3)
\psset{unit=0.5}
\psaxes[labels=none]{->}(11,5)[$x$,-90][$y$,180]
\psline{-}(1,1)(9,1)
\psline{-}(9,1)(9,3)
\psline{-}(9,3)(1,3)
\psline{-}(1,3)(1,1)
\end{pspicture}
\end{figure}

As you can see, the figure has a height of 2 and a length of 8, making its area 16 square units. Now, for the triangle to have an area of 16, the base could be 4 units long and the hight could be 8 units. The formula for the area of a triangle is $A=\frac{1}{2}(b\times h)$. If we plug in our values, we get $A=\frac{1}{2}(4\times 8)$. Simplified, it is $A=\frac{1}{2}(32)$, or $A=16$, which is the same area as the rectangle. The triangle would look like this:

\begin{figure}[h]
\centering
\begin{pspicture}(0,-0.5)(5,5)
\psset{unit=0.5}
\psaxes[labels=none]{->}(6,10)[$x$,-90][$y$,180]
\psline{-}(1,1)(5,1)
\psline{-}(5,1)(3,9)
\psline{-}(3,9)(1,1)
\end{pspicture}
\end{figure}

\section{Problem C}

Laurie solves the inequality below in the following way:

\begin{align*}
mx+3\le& 8\\
mx\le&5\\
x\le&\frac{5}{m}
\end{align*}
Will Laurie's solution be correct for all values of $m$? Explain your reasoning.

In short, no. If $m=0$, then you would have $x\le \frac{5}{0}$, which is undefined. For positive numbers, it obviously works, but for negative numbers, it does not. If we use $m=-1$ for her simplified equation, we get $x\le\frac{5}{-1}$, or $x\le-5$. However, if we put it in to the original equation, we get $-1x+3\le 8$, which simplifies to $-x\le 5$, which is $x\ge-5$. Because $x\le-5$ is not the same as $x\ge-5$, then using negative numbers in the equation does not work.

\end{document}