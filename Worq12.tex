\documentclass[a4paper]{article}

\usepackage[english]{babel} 
\usepackage[utf8]{inputenc}
\usepackage{amsmath}
\usepackage{graphicx}
\usepackage{float}
\usepackage[colorinlistoftodos]{todonotes}
\usepackage{pst-all}
\usepackage{caption}
\usepackage{float}
\usepackage{fullpage}

\title{WORQ 12}

\author{Theo Makler}

\date{\today}

\begin{document}
\maketitle

\section{Problem A}

Draw and give the dimensions of a rectangle and a triangle that have the same area. Explain why the two figures have the same area.

To start, I made a rectangle with an area of 16 square units. It looks like this:

\begin{figure}[h]
\centering
\begin{pspicture}(0,-0.5)(5,3)
\psset{unit=0.5}
\psaxes[labels=none]{->}(11,5)[$x$,-90][$y$,180]
\psline{-}(1,1)(9,1)
\psline{-}(9,1)(9,3)
\psline{-}(9,3)(1,3)
\psline{-}(1,3)(1,1)
\end{pspicture}
\end{figure}

As you can see, the figure has a hight of 2 and a length of 8, making its area 16 square units. Now, for the triangle to have an area of 16, the base could be 4 units long and the hight could be 8 units. The formula for the area of a triangle is $A=\frac{1}{2}(b\times h)$. If we plug in our values, we get $A=\frac{1}{2}(4\times 8)$. Simplified, it is $A=\frac{1}{2}(32)$, or $A=16$, which is the same area as the rectangle. The triangle would look like this:

\begin{figure}[h]
\centering
\begin{pspicture}(0,-0.5)(5,5)
\psset{unit=0.5}
\psaxes[labels=none]{->}(6,10)[$x$,-90][$y$,180]
\psline{-}(1,1)(5,1)
\psline{-}(5,1)(3,9)
\psline{-}(3,9)(1,1)
\end{pspicture}
\end{figure}

\section{Problem C}

Give the coordinates of the vertices of a non-square rectangle whose sides are not parallel to the axes. Explain why your vertices satisfy the conditions.

To make a rectangle whose sides are not parallel to the axes, I can make the sides have the slopes of 1 and $-1$. Also, to be a rectangle, two sides have to be longer than the other two sides. To solve this, I made the short sides go over 1 and up 1, and the long sides go over 2 and down 2. These made me pick the coordinates $A=(0,2)$, $B=(2,0)$, $C=(3,1)$ and $D=(1,3)$.

To show that these points are in a rectangle that satisfies the conditions, I can prove that there are two sets of parallel lines, there is a right angle, and that the short side is shorter than the long side. The equation for the parallel lines is this:

Line $AB$'s slope:
$$\frac{0-2}{2-0}=-1$$
Line $CD$'s slope:
$$\frac{3-1}{1-3}=-1$$
Line $AD$'s slope:
$$\frac{3-2}{1-0}=1$$
Line $BC$'s slope:
$$\frac{1-0}{3-2}=1$$

As these show, lines $AB$ and lines $CD$ are parallel, and lines $AD$ and $BC$ are parallel. This shows that two lines are perpendicular:

Line$AB$'s slope: $-1$
Line$AC$'s slope:$1$
$-1$ is the opposite reciprocal of $1$, making lines $AB$ and $AC$ perpendicular.

Now that I have shown that the figure is a rectangle, I need to show that it satisfies the conditions:

Length of line $AB$: 
$$a^2+b^2=c^2$$
$$2^2+2^2=c^2$$
$$4+4=c^2$$
$$8=c^2$$
$$\sqrt{8}=c$$

Length of line $AC$
$$a^2+b^2=c^2$$
$$1^2+1^2=c^2$$
$$1+1=c^2$$
$$2=c^2$$
$$\sqrt{2}=c$$

These equations show that line $AB$ is longer than line $AC$, making the rectangle fit the given requirements. The diagram is shown here:

\begin{figure}[h]
\centering
\begin{pspicture}(0,-0.5)(5,6)
\psset{unit=0.5}
\psaxes[labels=none]{->}(11,11)[$x$,-90][$y$,180]
\psline{-}(0,2)(2,0)
\psline{-}(3,1)(2,0)
\psline{-}(1,3)(3,1)
\psline{-}(1,3)(0,2)
\rput(-2,2){\psframebox*{$A(0,2)$}}
\rput(2,-1){\psframebox*{$B(2,0)$}}
\rput(5,1){\psframebox*{$C(3,1)$}}
\rput(2,3.5){\psframebox*{$D(1,3)$}}
\psdot(0,2)
\psdot(2,0)
\psdot(1,3)
\psdot(3,1)
\end{pspicture}
\end{figure}


\end{document}