\documentclass[a4paper]{article}

\usepackage[english]{babel} 
\usepackage[utf8]{inputenc}
\usepackage{amsmath}
\usepackage{graphicx}
\usepackage{float}
\usepackage[colorinlistoftodos]{todonotes}
\usepackage{pst-all}
\usepackage{caption}
\usepackage{float}
\usepackage{fullpage}

\title{WORQ 17}

\author{Theo Makler}

\date{\today}

\begin{document}
\maketitle

\section{Problem b}

You cannot conclude that Rectangle 1 has a larger area than Rectangle 2.

To prove this, I started with a rectangle with a relatively small area. I made the rectangle 1 unit by 24 units so it had an area of 24 square units and a perimeter of 50 units. I then made a second rectangle which had the dimensions 7 units by 8 units. This has an area of 56 square units and a perimeter of 30 units. Because the second rectangle had a larger area but a smaller perimeter, you cannot conclude that a rectangle with a larger perimeter will always have a larger area.

\section{Problem C}

The values of $a=2$, $b=3$, and $c=108$ used in the problem $a(x+b)^3=c$ make $x$ and integer.

To solve this, I first made $x=3$ because I know that 3 is a real number. Then, I chose values for $a$ and $b$ so I could solve for $c$:

$$a(x+b)^3=c$$
$$2(3+3)^3=c$$
$$2(27+27)=c$$
$$2\times54=c$$
$$108=c$$

As you can see, I used plugged in the variables to find that $c=108$. To make sure that my variables were correct, I plugged all the variables back in except for $x$ to make sure that I got an integer:

$$2(x+3)^3=108$$
$$2(x^3+27)=108$$
$$2x^3+54=108$$
$$2x^3=54$$
$$x^3=27$$
$$x=3$$

As these equations show, the variables $a=2$, $b=3$, and $c=108$ in $a(x+b)^3=c$ make $x$ an integer.

\end{document}