\documentclass[a4paper]{article}

\usepackage[english]{babel} 
\usepackage[utf8]{inputenc}
\usepackage{amsmath}
\usepackage{graphicx}
\usepackage{float}
\usepackage[colorinlistoftodos]{todonotes}
\usepackage{pst-all}
\usepackage{caption}
\usepackage{float}
\usepackage{fullpage}

\title{WORQ 17}

\author{Theo Makler}

\date{\today}

\begin{document}
\maketitle

\section{Problem b}

You cannot conclude that Rectangle 1 has a larger area than Rectangle 2.

To solve this problem, I started with a rectangle with a relatively small area. I made the rectangle 1 unit by 24 units so it had an area of 24 square units and a perimeter of 50 units. I then made a second rectangle which had the dimensions 7 units by 8 units. This has an area of 56 square units and a perimeter of 30 units. Because $30\ge 28$ and $50\ge 24$, you cannot say that a rectangle with a larger perimeter always has a larger area.

\section{Problem C}

The system of linear equations that has the solution $(3,-2,7)$ is $x-y+z=12$, $(x+y)\times z =(\frac{27}{9}\times\frac{1.5}{4.5})^\frac{1}{3}\times7$, and $\frac{x}{yz}\times\frac{-14}{3}=(\frac{1}{3}x \times (36^\frac{1}{2}) \div 6)$

To make these equations, I just plugged in the variables for $x$, $y$, and $z$ to make an equation that was true. To prove that these equations are correct, we can plug in the variables again.
$$x-y+z=12$$
$$3-(-2)+7=12$$
$$12=12$$
As you can see, I plugged the variables back in and proved that the equation is true. Next, I solve the second equation:
$$(x+y)\times z =(\frac{27}{9}\times\frac{1.5}{4.5})^\frac{1}{3}\times7$$
$$(3+(-2))\times 7 =(\frac{27}{9}\times\frac{1.5}{4.5})^\frac{1}{3}\times7$$
$$1\times 7 =1^\frac{1}{3}\times7$$
$$7=1\times7$$
$$7=7$$
The second equation is proven true. Finally, I solve the last equation:
$$\frac{x}{yz}\times\frac{-14}{3}=(\frac{1}{3}x \times (36^\frac{1}{2}) \div 6)$$
$$\frac{3}{-2\times7}\times\frac{-14}{3}=(\frac{1}{3}\times3 \times (36^\frac{1}{2}) \div 6)$$
$$1=1 \times 6 \div 6$$
$$1=1\times1$$
$$1=1$$
As you can see, the linear equations above make the solution $(3,-2,7)$

\end{document}