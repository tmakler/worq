\documentclass[a4paper]{article}

\usepackage[english]{babel} 
\usepackage[utf8]{inputenc}
\usepackage{amsmath}
\usepackage{graphicx}
\usepackage{float}
\usepackage[colorinlistoftodos]{todonotes}
\usepackage{pst-all}
\usepackage{caption}
\usepackage{float}
\usepackage{fullpage}

\title{WORQ 20}

\author{Problem C}

\date{\today}

\begin{document}
\maketitle

\section{Problem C}

If both $p$ and $q$ are integers, the value of $p^q$ will always be an integer.

Any integer multiplied by an integer is, by definition, an integer.

$$2\times3=6$$

In the problem, $q$ dictates how many times $p$ is multiplied by itself. The equation for $p^q$ would look like $p\times p\times p\times p$.... exactly $q$ times. If $p$ is an integer, no matter how small or large $q$ is, as long as $q$ is an integer, the solution to $p^q$ will be an integer.

\end{document}