\documentclass[a4paper]{article}

\usepackage[english]{babel} 
\usepackage[utf8]{inputenc}
\usepackage{amsmath}
\usepackage{graphicx}
\usepackage{float}
\usepackage[colorinlistoftodos]{todonotes}
\usepackage{pst-all}
\usepackage{caption}
\usepackage{float}
\usepackage{fullpage}

\title{WORQ 8}

\author{Theo Makler}

\date{\today}

\begin{document}
\maketitle

\section{Problem B}

Given that $\frac{x}{2}=\frac{y}{4}$, Phillip claims that $\frac{x+1}{2}=\frac{y+1}{4}$ because he added the same value to both sides. Is his claim correct? Why or why not?

As a start, I plugged in values for $x$ and $y$ that made the first equation true. I came up with $\frac{1}{2}=\frac{2}{4}$. If we use these values in the second equation, we get $\frac{1+1}{2}=\frac{2+1}{4}$, or $\frac{2}{2}=\frac{3}{4}$, which is not true. 

If we examine the question carefully, we see that Phillip added $\frac{1}{2}$ to the left side and $\frac{1}{4}$ to the right side as you can see here:

$$\frac{x+1}{2}=\frac{x}{2}+\frac{1}{2}$$

$$\frac{y+1}{4}=\frac{y}{4}+\frac{1}{4}$$

Because $\frac{1}{2}$ does not equal $\frac{1}{4}$, then Phillip did not add the same value to both sides, making his statement false.

\section{Problem C}

Choose values for $a$ and $b$ so that $(4,-3)$, $(-2,b)$, and $(a,7)$ are collinear. Show that your points are collinear.

For these three points to be collinear, they need to be in a line. If we graph $(4,-3)$ and all possible points for $(a,7)$ and $(-2,b)$ on a graph, they would look like this:

\begin{figure}[h]
\centering
\begin{pspicture}(-3,-3)(5,5)
\psset{unit=0.5}
\psaxes[labels=none]{<->}(0,0)(-11,-5)(8,9)[$x$,-90][$y$,180]
\psdot(4,-3)
\rput(4,-4){\psframebox*{$(4, -3)$}}
\psline[linestyle=dashed,dash=3pt 2pt, linecolor=blue]{<->}(-2,-5)(-2,8)
\rput(-2,-3){\psframebox*{\blue $(-2,b)$}}
\psline[linestyle=dashed,dash=3pt 2pt, linecolor=red]{<->}(-11,7)(6,7)
\rput(3,7){\psframebox*{\red $(a,7)$}}
\end{pspicture}
\end{figure}

This graph shows the point $(4,-3)$ and all possible points for $(-2,b)$ and $(a,7)$. If we draw a line from $(4,-3)$ that touches these points with a slope of $-1$, it would look like this:

\begin{figure}[h]
\centering
\begin{pspicture}(-3,-3)(5,5)
\psset{unit=0.5}
\psaxes[labels=none]{<->}(0,0)(-11,-5)(8,9)[$x$,-90][$y$,180]
\psdot(4,-3)
\rput(3,-4){\psframebox*{$(4, -3)$}}
\psline[linestyle=dashed,dash=3pt 2pt, linecolor=blue]{<->}(-2,-5)(-2,8)
\rput(-2,-3){\psframebox*{\blue $(-2,b)$}}
\psline[linestyle=dashed,dash=3pt 2pt, linecolor=red]{<->}(-11,7)(6,7)
\rput(3,7){\psframebox*{\red $(a,7)$}}
\psline{<->}(5,-4)(-7,8)
\psdot(-2,3)
\rput(-3.3,2.25){\psframebox*{$(-2,3)$}}
\psdot(-6,7)
\rput(-5,8){\psframebox*{$(-6,7)$}}
\end{pspicture}
\end{figure}

From this graph, we can see the points at which the line intersects, which are $(-2,3)$ and $(-6,7)$. To show that these points are on the line that we drew passing through $(4,-3)$, we can plug them into the equation for that line:

$$x+y=1$$

If $x=-2$ and $y=3$, then this equation is true:

$$-2+3=1$$

The equation is also true if we use $x=-6$ and $y=7$:

$$-6+7=1$$

Therefore, if $a=-6$ and $b=3$, then all three points $(a,7)$, $(-2,b)$ and$(4,-3)$ are collinear along the line $x+y=1$. 

\end{document}