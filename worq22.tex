\documentclass[a4paper]{article}

\usepackage[english]{babel} 
\usepackage[utf8]{inputenc}
\usepackage{amsmath}
\usepackage{graphicx}
\usepackage{float}
\usepackage[colorinlistoftodos]{todonotes}
\usepackage{pst-all}
\usepackage{caption}
\usepackage{float}
\usepackage{fullpage}

\title{WORQ 22}

\author{Theo Makler}

\date{\today}

\begin{document}
\maketitle

\section{Problem A}

The equation $y=-x-5$ creates a line that goes through the point $(0,-5)$ and contains points in the second quadrant.

To solve this problem, I made the $y$-intercept $-5$ so that the line contained the point $(0,-5)$. So far, the graph looks like this:

\begin{figure}[h]
\centering
\begin{pspicture}(.25,2)(.25,-2)
\psset{unit=.3}
\psaxes[labels=none]{->}(0,0)(-7,-7)[$x$,-90][$y$,180]
\psaxes[labels=none]{->}(0,0)(7,7)[$x$,-90][$y$,180]
\psdot(0,-5)
\end{pspicture}
\end{figure}

Now, I made a line with a slope of $-1$ so that the line went through the second quadrant.

\begin{figure}[h]
\centering
\begin{pspicture}(.25,2)(.25,-2)
\psset{unit=.3}
\psaxes[labels=none]{->}(0,0)(-7,-7)[$x$,-90][$y$,180]
\psaxes[labels=none]{->}(0,0)(7,7)[$x$,-90][$y$,180]
\psdot(0,-5)
\psline{-}(2,-7)(-7,2)
\end{pspicture}
\end{figure}

As you can see, the line  $y=-x-5$ creates a line that goes through the point $(0,-5)$ and contains points in the second quadrant.

\pagebreak

\section{Problem C}

The coordinates $a(-5,5)$ and $b(5,-5)$ make the line segment $ab$ run through the origin. 

To make the problem relatively easier, I made the line segment $ab$'s slope $-1$ so the points would be easy to find. I first plotted the line on the graph.

\begin{figure}[h]
\centering
\begin{pspicture}(.25,2)(.25,-2)
\psset{unit=.3}
\psaxes[labels=none]{->}(0,0)(-7,-7)[$x$,-90][$y$,180]
\psaxes[labels=none]{->}(0,0)(7,7)[$x$,-90][$y$,180]
\psline{-}(-7,7)(7,-7)
\end{pspicture}
\end{figure}

I knew that if I picked a point from Quadrant II and a point from Quadrant IV on this line, the points would satisfy the conditions. I picked $(-5,5)$ and $(5,-5)$.

\begin{figure}[h]
\centering
\begin{pspicture}(.25,2)(.25,-2)
\psset{unit=.3}
\psaxes[labels=none]{->}(0,0)(-7,-7)[$x$,-90][$y$,180]
\psaxes[labels=none]{->}(0,0)(7,7)[$x$,-90][$y$,180]
\psline{-}(-5,5)(5,-5)
\psdot(5,-5)
\psdot(-5,5)
\end{pspicture}
\end{figure}

As this shows, the points $a(-5,5)$ and $b(5,-5)$ make the line segment $ab$ run through the origin.

\section{Problem D}

The values $a=-5$ and $m=3$ make the expression $a^m$ less than 0.

Starting out, I knew that a negative number multiplied an odd number of times will create a negative number. For this problem, I just have to make $a$ negative and $m$ odd for $a^m$ to be negative. I chose $a=-5$ and $m=3$.
$$a^m$$
$$(-5)^3$$
$$-125$$
These equations show that the values $a=-5$ and $m=3$ used in the expression $a^m$ make the expression negative.









\end{document}