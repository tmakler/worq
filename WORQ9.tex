\documentclass[a4paper]{article}

\usepackage[english]{babel} 
\usepackage[utf8]{inputenc}
\usepackage{amsmath}
\usepackage{graphicx}
\usepackage{float}
\usepackage[colorinlistoftodos]{todonotes}
\usepackage{pst-all}
\usepackage{caption}
\usepackage{float}
\usepackage{fullpage}

\title{WORQ 9}

\author{Theo Makler}

\date{\today}

\begin{document}
\maketitle

\section{Problem A}

The price of a necklace was first increased by $50\%$ and later decreased by $50\%$. Is the final price the same as the original price? Why or why not?

To represent the price of the necklace, we can make it variable $a$. To show the price increasing and decreasing, we can show this equation:

$$\left(a+.5a\right)\times.5$$

To solve the equation, we simplify like so:

$$1.5a\times.5$$

$$.75a$$

As these equations show, raising the price by $50\%$ and then lowering the price by $50\%$ makes the price $75\%$ of the original.

\section{Problem C}

Give the coordinates of point $D$ so that $ABCD$ is a parallelogram that is not a rhombus. Explain why the coordinates of point $D$ satisfy the conditions or argue that is is impossible to find such a point.

To begin, I drew the graph with given information:

\begin{figure}[h]
\centering
\begin{pspicture}(-3,-3)(5,5)
\psset{unit=0.5}
\psaxes[labels=none]{<->}(0,0)(-7,-5)(8,9)[$x$,-90][$y$,180]
\psdot(0,0)
\rput(-2,1){\psframebox*{$Point A$}}
\psdot(6,0)
\psline{-}(0,0)(3,-4)
\psdot(3,-4)
\psline{-}(3,-4)(6,0)
\rput(3,-5){\psframebox*{$Point B$}}
\rput(6,1){\psframebox*{$Point C$}}
\end{pspicture}
\end{figure}

Given this information, is is impossible to find a point $D$ that makes the figure a parallelogram and not a rhombus because the line from point $A$ to point $B$ and the line from point $B$ to point $C$ are the same length. If we draw a vertical line from point $B$ at $(3,-4)$, then we see that there are two triangles like so:

\begin{figure}[h]
\centering
\begin{pspicture}(-3,-3)(5,5)
\psset{unit=0.5}
\psaxes[labels=none]{<->}(0,0)(-7,-5)(8,9)[$x$,-90][$y$,180]
\psdot(0,0)
\rput(-2,1){\psframebox*{$Point A$}}
\psdot(6,0)
\psline{-}(0,0)(3,-4)
\psdot(3,-4)
\psline{-}(3,-4)(6,0)
\rput(3,-5){\psframebox*{$Point B$}}
\rput(6,1){\psframebox*{$Point C$}}
\psline{-}(3,-4)(3,0)
\end{pspicture}
\end{figure}

If we use the Pythagorean Theorem, we can find the lengths of the lines $AB$ and $BC$:

$$a^2+b^2=c^2$$
$$3^2+4^2=c^2$$
$$9+16=c^2$$
$$25=c^2$$
$$5=c$$

If $ABCD$ were a parallelogram, then $AD$ would have to be the same length as $BC$, and $CD$ would have to be the same length as $AB$.
Because the two segments $AB$ and $BC$ are the same length, this would mean that all four sides had the same length, making the figure a rhombus.

\end{document}