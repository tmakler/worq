\documentclass[a4paper]{article}

\usepackage[english]{babel} 
\usepackage[utf8]{inputenc}
\usepackage{amsmath}
\usepackage{graphicx}
\usepackage{float}
\usepackage[colorinlistoftodos]{todonotes}
\usepackage{pst-all}
\usepackage{caption}
\usepackage{float}
\usepackage{fullpage}

\title{WORQ 5}

\author{Theo Makler}

\date{\today}

\begin{document}
\maketitle

\section{Problem A}

Nicole ate $\frac{1}{3}$ of a pizza. Mike ate $\frac{1}{3}$ of the remaining pizza. Dan concluded that there was $\frac{1}{3}$ of the pizza left for him. Is Dan's conclusion correct? Why or why not?

To figure out this problem, I thought about the pizza. If $\frac{1}{3}$ of the pizza had been eaten, then there must be $\frac{2}{3}$ remaining. If Mike ate $\frac{1}{3}$ of the remaining, that means that he would be eating $\frac{1}{3}$ of $\frac{2}{3}$.
$$\frac{1}{3}\times\frac{2}{3}=\frac{2}{9}$$
This shows that Mike is eating $\frac{2}{9}$ of the total pizza, not $\frac{1}{3}$. To find the ammount of pizza that Dan gets, you have to add together the pizza that Nicole and Mike ate.
$$ \frac{1}{3}+ \frac{2}{9}= \frac{5}{9}$$
This equation shows that Nicole and Mike ate $\frac{5}{9}$ of the total pizza, which means that Dan will get $\frac{4}{9}$ of the pizza, proving that Dan's conclusion was incorrect.

\section{Problem C}

In this problem, you have the equation $\frac{a}{b}\times\frac{c}{d}=\frac{7}{12}$, and you are supposed to find values for $a$, $b$, $c$, and $d$ that make the equation true and are not 1.

I know that $\frac{a}{b}\times\frac{c}{d}$ does not have to make exactly $\frac{7}{12}$, but it can make an equivalent fraction, like $\frac{14}{24}$. From here, I tried different solutions for $a$, $b$, $c$, and $d$. Eventualy, I came up with this:
$$\frac{14}{6}\times\frac{2}{8}=\frac{7}{12}$$
From this, you can smplify the fractions to this:
$$\frac{28}{48}=\frac{7}{12}$$
Which can be further simplifyed to this:
$$\frac{7}{12}=\frac{7}{12}$$
For this example, the solutions for $a$, $b$, $c$, and $d$ are $a=14$, $b=6$, $c=2$, and $d=8$.

\pagebreak
\section{Problem D}

In this problem, you have to find a data set of 10 numbers whose mean is 20, median is 18, and mode is 25.

To start, I made a chart like this:
$$a,b,c,d,e,f,g,h,i,j$$
I know that the median is 18, so I can replace $e$ and $f$ with 18.
$$a,b,c,d,18,18,g,h,i,j$$
I also know that the mode is 25, so I can replace $g$, $h$, $i$, and $j$ with 25.
$$a,b,c,d,18,18,25,25,25,25$$
Finally, I know that the mean, or average, is 20, so I did this:
\begin{align*}
(a+b+c+d+18+18+25+25+25+25)\times\frac{1}{10}&=20\\
(a+b+c+d+18+18+25+25+25+25)&=200\\
a+b+c+d+136&=200\\
a+b+c+d&=64
\end{align*}
Now that I know that $a+b+c+d=64$, I just need to find numbers that fit into the slots that I have made.
\begin{align*}
a+b+c+d&=64\\
15+15+17+17&=64\\
64&=64
\end{align*}
It would work to have $a$, $b$, $c$, and $d$ all equal 16, but then there would be two modes because there would be four 16's and four 25's. Therefore, the average of $a$, $b$, $c$, and $d$ must equal 16 but cannot consist entirely of 16's. For this problem, I used $15,15,17,17$, but another solution could be $14,15,17,18$.

From these equations, I can say that my solution set that has a median of 18, mode of 25, and mean of 20 is this:
$$15,15,17,17,18,18,25,25,25,25$$

\end{document}