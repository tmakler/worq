\documentclass[a4paper]{article}

\usepackage[english]{babel} 
\usepackage[utf8]{inputenc}
\usepackage{amsmath}
\usepackage{graphicx}
\usepackage{float}
\usepackage[colorinlistoftodos]{todonotes}
\usepackage{pst-all}
\usepackage{caption}
\usepackage{float}
\usepackage{fullpage}

\title{WORQ 10}

\author{Theo Makler}

\date{\today}

\begin{document}
\maketitle

\section{Problem }

In the candle problem, there are two candles of equal length $l$. One candle $y_{1}$ burns out in 9 hours, and the other candle $y_{2}$ burns out in 6 hours. At what time $t$ is $y_{1}$ twice as long as $y_{2}$?

To figure out this problem, I first made a graph like Figure 1 with time on the $x$ axis and the length of each candle on the $y$ axis. fart

\begin{figure}[h]
\centering
\begin{pspicture}(0,-0.5)(5,6)
\psset{unit=0.5}
\psaxes[labels=none]{->}(11,11)[$x$,-90][$y$,180]
\psline{-}(0,9)(9,0)
\psline{-}(0,9)(6,0)
\psline[linestyle=dashed,dash=3pt 2pt](4.5,0)(4.5,4.5)
\psline[linestyle=dashed,dash=3pt 2pt](0,4.5)(4.5,4.5)
\psline[linestyle=dashed,dash=3pt 2pt](0,2.25)(4.5,2.25)
\rput(-0.4,9){\psframebox*{$l$}}
\rput(-0.6,4.5){\psframebox*{$y_{1}$}}
\rput(-0.6,2.25){\psframebox*{$y_{2}$}}
\rput(4.5,-0.6){\psframebox*{$t$}}
\rput(6,-0.6){\psframebox*{$6$}}
\rput(9,-0.6){\psframebox*{$9$}}
\end{pspicture}
\caption{The Initial Setup}
\end{figure}

The line that goes from $(0,l)$ to $(9,0)$ is the slower burning candle and the line that goes from $(0,l)$ to $(6,0)$ is the faster burning candle.

To write the equation for the lines in the graph, you need two peices of information: the $y$-intercept and the slope of the line. In the case of the line $y_{1}$, the $y$-intercept is $l$ because at time 0, the candle is at its full length, and the slope is $-\frac{l}{9}$ because the rise is $-l$ and the run is 9. Therefore, the equation for the top line can be written:
$$y_{1}=l-\frac{l}{9}$$
The equation for the second line is virtually the same, written as so:
$$y_{2}=l-\frac{l}{6}$$

To to solve this problem, I need to find a $t$ where $y_{1}$ is twice as much as $y_{2}$:
$$y_{1} = 2 \times y_{2}$$
Then, I substituted the equation for each line in for the $y_{1}$ and $y_{2}$ to make this:
$$l-\frac{l}{9}t = 2 \left(l-\frac{l}{6}t\right)$$
This equation represents the two equation for the lines, with a given hight of $l$. After this, I divided both sides by $l$ like so:
$$1-\frac{1}{9}t = 2 \left(1-\frac{1}{6}t\right)$$
From here, it was a process of solving a simple equation.
\begin{align*}
1-\frac{1}{9}t &= 2-\frac{1}{3}t &&\text{Distribute the 2.}\\
9\times\left(1-\frac{1}{9}t\right) &= \left(2-\frac{1}{3}t\right)\times9 &&\text{Multiply each side by 9.}\\ 
9-t &= 18-3t &&\text{Add $3t$ to each side.}\\
9+2t &= 18 &&\text{Subract 9 from each side.}\\
2t&=9 &&\text{Divide each side by 2.}\\
t&=4.5
\end{align*}

As this equation shows, the slower burning candle will be twice as long as the faster burning candle after 4.5 hours.

\section{The Bonus}

In the Bonus problem, there are two candles of equal length $l$. One candle $y_{1}$ burns out in 9 hours, and the other candle $y_{2}$ burns out in $T$ hours. At what time $t$ is $y_{1}$ twice as long as $y_{2}$?

As with the first problem, I first made a graph to show the problem:

\begin{figure}[h]
\centering
\begin{pspicture}(0,-0.5)(5,6)
\psset{unit=0.5}
\psaxes[labels=none]{->}(11,11)[$x$,-90][$y$,180]
\psline{-}(0,9)(9,0)
\psline{-}(0,9)(6,0)
\psline[linestyle=dashed,dash=3pt 2pt](4.5,0)(4.5,4.5)
\psline[linestyle=dashed,dash=3pt 2pt](0,4.5)(4.5,4.5)
\psline[linestyle=dashed,dash=3pt 2pt](0,2.25)(4.5,2.25)
\rput(-0.4,9){\psframebox*{$l$}}
\rput(-0.6,4.5){\psframebox*{$y_{1}$}}
\rput(-0.6,2.25){\psframebox*{$y_{2}$}}
\rput(4.5,-0.6){\psframebox*{$t$}}
\rput(6,-0.6){\psframebox*{$T$}}
\rput(9,-0.6){\psframebox*{$9$}}
\end{pspicture}
\caption{The Initial Setup}
\end{figure}

To solve this problem, I took the equation from part 1 and replaced the 6 with $T$:

$$l-\frac{l}{9}t = 2 \left(l-\frac{l}{T}t\right)$$

Now, I just had to solve this equation for the variable $t$.
\begin{align*}
1-\frac{1}{9}t &= 2-\frac{2}{T}t &&\text{Distribute the 2.}\\
9\times\left(1-\frac{1}{9}t\right) &= \left(2-\frac{2}{T}t\right)\times9 &&\text{Multiply each side by 9.}\\ 
9-t &= 18-\frac{18}{T}t &&\text{Add $\frac{18}{T}t$ to each side.}\\
9+\frac{18}{T}t -t&= 18 &&\text{Subract 9 from each side.}\\
\left(\frac{18}{T}-1\right)t&=9 &&\text{Divide each side by $\frac{18}{T}-1$.}\\
t&=\frac{9}{\frac{18}{T}-1} 
\end{align*}
As this equation shows, candle $y_{1}$ will be twice as long as candle $y_{2}$ after $\frac{9}{\frac{18}{T}-1}$ hours.
To check this answer, we can plug 6 back in for $T$ to see if we get 4.5:

\begin{align*}
t&=\frac{9}{\frac{18}{6}-1}\\
t&=\frac{9}{3-1}\\
t&=\frac{9}{2}\\
t&=4.5
\end{align*}
\end{document}