\documentclass[a4paper]{article}

\usepackage[english]{babel} 
\usepackage[utf8]{inputenc}
\usepackage{amsmath}
\usepackage{graphicx}
\usepackage{float}
\usepackage[colorinlistoftodos]{todonotes}
\usepackage{pst-all}
\usepackage{caption}
\usepackage{float}
\usepackage{fullpage}

\title{WORQ 10}

\author{Theo Makler}

\date{\today}

\begin{document}
\maketitle

\section{Problem B}

Create a unit of length so that the pencil above would be more than 2 units long but less than three units long. Explain how you decided on the length of you unit.
(The pencil is 3$\frac{1}{2}$ '' long)

First, I decided that I want to make the pencil 2$\frac{1}{2}$ of my unit because that is between 2 and 3. I know that the pencil is 3$\frac{1}{2}$ '' long, so I can write an equation to show how long one of my units is in inches.
$$3\frac{1}{2}=2\frac{1}{2}x$$
If I solve for $x$, then I will find the length of my unit in inches.
$$\frac{7}{2}=\frac{5}{2}x$$
$$\frac{7}{2}\times\frac{2}{5}=x$$
$$\frac{7}{5}=x$$
$$1.4=x$$
From these equations, I have found that each of my units is 1.4 inches long. 2$\frac{1}{2}$ of these units is the length of the pencil.

\section{Problem C}

Give the coordinates of the vertices of an isosceles triangle whose axis of symmetry is not parallel to either axis. Show that your triangle is isosceles.

I know that if I make an isosceles triangle whose base has a slope of 1, the axis of symmetry will not be parallel to either axis. Because of this, I started with a base that went from $(1,1)$ to $(5,5)$. The axis of symmetry has to be perpendicular to the base (i.e. have a slope of $-1$) and has to pass through the midpoint, which is $(3,3)$. Using point-slope formula, the equation of the line is :

$$y-3=-1(x-3)$$
$$y-3=-x+3$$
$$y=-x+6$$

Any point along the axis of symmetry (except $(3,3)$) will make an isosceles triangle with $(1,1)$ and $(5,5)$. I chose $(0,6)$ because it is the y-intercept. 

\begin{figure}[h]
\centering
\begin{pspicture}(0,-0.5)(5,6)
\psset{unit=0.5}
\psaxes[labels=none]{->}(11,11)[$x$,-90][$y$,180]
\psline{-}(1,1)(5,5)
\psline[linestyle=dashed,dash=3pt 2pt](6,0)(-3,9)
\psline{-}(0,6)(1,1)
\psline{-}(0,6)(5,5)
\rput(-2,6){\psframebox*{$A(0,6)$}}
\rput(3,1){\psframebox*{$B(1,1)$}}
\rput(5,6){\psframebox*{$C(5,5)$}}
\end{pspicture}
\end{figure}

The answer for these points is $A=(0,6)$, $B=(1,1)$ and $C=(5,5)$. 

\end{document}