\documentclass[a4paper]{article}

\usepackage[english]{babel} 
\usepackage[utf8]{inputenc}
\usepackage{amsmath}
\usepackage{graphicx}
\usepackage{float}
\usepackage[colorinlistoftodos]{todonotes}
\usepackage{pst-all}
\usepackage{caption}
\usepackage{float}
\usepackage{fullpage}

\title{WORQ 10}

\author{Theo Makler}

\date{\today}

\begin{document}
\maketitle

\section{Problem B}

Create a unit of length so that the pencil above would be more than 2 units long but less than three units long. Explain how you decided on the length of you unit.
(The pencil is 3$\frac{1}{2}$ '' long)

First, I decided that I want to make the pencil 2$\frac{1}{2}$ of my unit because that is between 2 and 3. I know that the pencil is 3$\frac{1}{2}$ '' long, so I can write an equation to show how long one of my units is in inches.
$$3\frac{1}{2}=2\frac{1}{2}x$$
If I solve for $x$, then I will find the length of my unit in inches.
$$\frac{7}{2}=\frac{5}{2}x$$
$$\frac{7}{2}\times\frac{2}{5}=x$$
$$\frac{7}{5}=x$$
$$1.4=x$$
From these equations, I have found that each of my units is 1.4 inches long. 2$\frac{1}{2}$ of these units is the length of the pencil.

\section{Problem C}

Give the coordinates of the vertices of and isosceles triangle whose axis of symmetry is not parallel to either axis. Show that your triangle is isosceles.

I know that if I make an isosceles triangle whose base has a slope of 1, the axis of symmetry will not be parallel to either axis. 

\begin{figure}[h]
\centering
\begin{pspicture}(0,-0.5)(5,6)
\psset{unit=0.5}
\psaxes[labels=none]{->}(11,11)[$x$,-90][$y$,180]
\psline{-}(1,1)(5,5)
\psline[linestyle=dashed,dash=3pt 2pt](3,3)(0,6)
\psline{-}(0,6)(1,1)
\psline{-}(0,6)(5,5)
\rput(-2,6){\psframebox*{$A(0,6)$}}
\rput(3,1){\psframebox*{$B(1,1)$}}
\rput(5,6){\psframebox*{$C(5,5)$}}
\end{pspicture}
\caption{The Initial Setup}
\end{figure}

To show that my coordinates satisfy the conditions, I can prove that the triangle is isosceles. If the axis of symmetry is perpendicular to the base, then the triangle is isosceles. To find this, we can find the slopes of the base and of the axis of symmetry.

This equation shows the slope of the axis of symmetry:
$$m=\frac{6-3}{0-3}$$
$$m=\frac{3}{-3}$$
$$m=-1$$

The slope of the axis of symmetry is -1. This is the equation for the slope of the base:
$$m=\frac{5-1}{5-1}$$
$$m=\frac{4}{4}$$
$$m=1$$

The slope of the base is 1. Because -1 is the opposite reciprocal of 1, the axis of symmetry is perpendicular to the base. This, and the fact that the lines $AB$ and $AC$ run from the same point on the axis of symmetry make the triangle $ABC$ an isosceles triangle.

\end{document}