\documentclass[a4paper]{article}

\usepackage[english]{babel} 
\usepackage[utf8]{inputenc}
\usepackage{amsmath}
\usepackage{graphicx}
\usepackage{float}
\usepackage[colorinlistoftodos]{todonotes}
\usepackage{pst-all}
\usepackage{caption}
\usepackage{float}
\usepackage{fullpage}

\title{WORQ 15}

\author{Theo Makler}

\date{\today}

\begin{document}
\maketitle

\section{Problem A}

Give values for $a$, $b$, and $c$ so that the volume of the right triangular prism is the same as the volume of a cube with side length 4. Show that the volume of your prism is the same as the volume of the cube.

To start, I found the volume of the cube. If the side length is 4, then the cube has a volume of $64^3$ units. Next, I wrote down the formula for the volume of a triangular prism, with $a$, $b$, and $c$ being the length, width, and height: 
$$\frac{1}{2} \left(a \times b \right) \times c$$

Now, if we want this to have the same volume as the cube, we make it equal to 64:
$$\frac{1}{2} \left(a \times b \right) \times c=64$$
From here, we just simplify:
$$ \left(a \times b \right) \times c = 128$$
$$a \times b \times c = 128$$

As this shows, as long as the side's product is equal to 128, the prism will have the same volume as the cube. For this, I will let $a=4$, $b=4$, and $c=8$. If we do $4\times4\times8$, we get 128.

\section{Problem C}

Right triangles $ABC$ and $DEF$ have right angles $B$ and $F$. If line $AC$ is congruent to line $FD$, is it possible that triangle $ABC$ is congruent to triangle $DEF$? Why or why not?

To start, I drew possible triangles, and labeled them with the known lengths, assuming that the triangles are identical.

\begin{figure}[h]
\centering
\begin{pspicture}(-3,-1)(5,3)
\psset{unit=0.5}
\psline{-}(-4,0)(0,0)
\psline{-}(0,0)(0,4)
\psline{-}(0,4)(-4,0)

\psline{-}(2,0)(6,0)
\psline{-}(6,0)(6,4)
\psline{-}(6,4)(2,0)

\rput(-4.5,0){\psframebox*{$A$}}
\rput(0,-.5){\psframebox*{$B$}}
\rput(0,4.5){\psframebox*{$C$}}
\rput(1.5,0){\psframebox*{$D$}}
\rput(6,4.5){\psframebox*{$E$}}
\rput(6,-0.5){\psframebox*{$F$}}

\rput(-2.5,3){\psframebox*{$x$}}
\rput(4,-0.5){\psframebox*{$x$}}
\end{pspicture}
\end{figure}

I have labeled these triangles as such because, if line $AC$ is the same length as line $DF$, it can be labeled with a variable. If the two triangles are congruent, then lines $AB$ and $DE$ must be the same as $AC$ and $DF$, so those can be labeled with the same variable. However, the lines $BC$ and $EF$ must be a different length, so they are labeled with a different variable like so:

\begin{figure}[h]
\centering
\begin{pspicture}(-3,-1)(5,3)
\psset{unit=0.5}
\psline{-}(-4,0)(0,0)
\psline{-}(0,0)(0,4)
\psline{-}(0,4)(-4,0)

\psline{-}(2,0)(6,0)
\psline{-}(6,0)(6,4)
\psline{-}(6,4)(2,0)

\rput(-4.5,0){\psframebox*{$A$}}
\rput(0,-.5){\psframebox*{$B$}}
\rput(0,4.5){\psframebox*{$C$}}
\rput(1.5,0){\psframebox*{$D$}}
\rput(6,4.5){\psframebox*{$E$}}
\rput(6,-0.5){\psframebox*{$F$}}

\rput(-2.5,3){\psframebox*{$x$}}
\rput(4,-0.5){\psframebox*{$x$}}
\rput(-2,-0.5){\psframebox*{$x$}}
\rput(3.5,3){\psframebox*{$x$}}
\rput(0.5,2){\psframebox*{$y$}}
\rput(6.5,2){\psframebox*{$y$}}
\end{pspicture}
\end{figure}

Using the Pythagorean Theorem, we can see that this configuration is impossible, because $x^2 + y^2 = x^2$, meaning that $y^2 = 0$. This would make the figure a line, not a triangle, rendering it impossible for triangles $ABC$ and $DEF$ to be congruent.



\end{document}