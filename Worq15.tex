\documentclass[a4paper]{article}

\usepackage[english]{babel} 
\usepackage[utf8]{inputenc}
\usepackage{amsmath}
\usepackage{graphicx}
\usepackage{float}
\usepackage[colorinlistoftodos]{todonotes}
\usepackage{pst-all}
\usepackage{caption}
\usepackage{float}
\usepackage{fullpage}

\title{WORQ 15}

\author{Theo Makler}

\date{\today}

\begin{document}
\maketitle

\section{Problem A}

Give values for $a$, $b$, and $c$ so that the volume of the right triangular prism is the same as the volume of a cube with side length 4. Show that the volume of your prism is the same as the volume of the cube.

To start, I found the volume of the cube. If the side length is 4, then the cube has a volume of $64^3$ units. Next, I wrote down the formula for the volume of a triangular prism, with $a$, $b$, and $c$ being the length, width, and height: 
$$\frac{1}{2} \left(a \times b \right) \times c$$

Now, if we want this to have the same volume as the cube, we make it equal to 64:
$$\frac{1}{2} \left(a \times b \right) \times c=64$$
From here, we just simplify:
$$ \left(a \times b \right) \times c = 128$$
$$a \times b \times c = 128$$

As this shows, as long as the side's product is equal to 128, the prism will have the same volume as the cube. For this, I will let $a=4$, $b=4$, and $c=8$. If we do $4\times4\times8$, we get 128.

\pagebreak

\section{Problem C}

Give a quadratic function and a non-constant linear function whose graphs do not intersect. Explain how you know that the graphs of your two functions do not intersect.

If I made a quadratic function that was $y=x^2$, then the graph would look like this:

\begin{figure}[h]
\centering
\begin{pspicture}(-3,-3)(5,5)
\psset{unit=0.5}
\savedata {\mydata}[
{{0,0}, {1,1}, {2,4}}]
\dataplot[plotstyle=curve] {\mydata}
\psaxes[labels=none]{<->}(0,0)(-6,-6)(7,7)[$x$,-90][$y$,180]
\end{pspicture}
\end{figure}

If I make a non-constant linear function that never intersects this graph, it could look like this:

\begin{figure}[h]
\centering
\begin{pspicture}(-3,-3)(5,5)
\psset{unit=0.5}
\psaxes[labels=none]{<->}(0,0)(-6,-6)(7,7)[$x$,-90][$y$,180]
\psline{-}(0,-1)(6,-3)
\psline{-}(0,-1)(-6,1)
\end{pspicture}
\end{figure}

If we put the two together, it would look like this:

\pagebreak

\begin{figure}[h]
\centering
\begin{pspicture}(-3,-3)(5,5)
\psset{unit=0.5}
\psaxes[labels=none]{<->}(0,0)(-6,-6)(7,7)[$x$,-90][$y$,180]
\psline{-}(0,-1)(6,-3)
\psline{-}(0,-1)(-6,1)
\end{pspicture}
\end{figure}

At point $(-3,0)$, the non-constant linear function it at its x-intercept. At $x=-3$, the quadratic function is at $(-3, 9)$. Because the y-values will get exponentially higher, and the linear function is below the quadratic function, the linear function will never intersect the quadratic function.
\end{document}