\documentclass[a4paper]{article}

\usepackage[english]{babel} 
\usepackage[utf8]{inputenc}
\usepackage{amsmath}
\usepackage{graphicx}
\usepackage{float}
\usepackage[colorinlistoftodos]{todonotes}
\usepackage{pst-all}
\usepackage{caption}
\usepackage{float}
\usepackage{fullpage}

\title{WORQ 21}

\author{Theo Makler}

\date{\today}

\begin{document}
\maketitle

\section{Problem B}

The coordinates $b=(0,0)$ and $c=(2,0)$ make the triangle $abc$ isosceles.

To make an isosceles triangle with the point $a=1,4$, I first knew that I could put the points $b$ and $c$ on the x-axis. I would need to evenly space them from the x-axis for point $a$ to make the triangle isosceles. The triangle looks like this:

\begin{figure}[h]
\centering
\begin{pspicture}(.25,3.5)(.25,.25)
\psset{unit=.8}
\psaxes[labels=none]{->}(4,5)[$x$,-90][$y$,180]
\psdot(0,0)
\psdot(2,0)
\psdot(1,4)
\psline{-}(1,4)(0,0)
\psline{-}(0,0)(2,0)
\psline{-}(2,0)(1,4)
\end{pspicture}
\end{figure}

As you can see, I have positioned points $b$ and $c$ in the coordinates $(0,0)$ and $(2,0)$ respectively to create an isosceles triangle with the point $(1,4)$

\section{Problem C}

The values $a=1$, $b=2$, and $c=4$ make the equation $a^(b^c)+1$ positive.

Starting the problem, I knew that any exponent of 1, no matter how big or small, would be 1, so I could choose any values of $b$ and $c$ that I wanted to. 

$$[1^(2^4)]+1$$
$$1+1=2$$

 Because 2 is an even number, possible values for $a$, $b$, and $c$ in the equation $a^(b^c)+1$ to make it positive would be  $a=1$, $b=2$, and $c=4$

\end{document}